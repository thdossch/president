\documentclass[11pt]{article}
\usepackage{a4wide}
\usepackage{float}
\usepackage{graphicx}
\usepackage{hyperref}
\usepackage[margin=1in,footskip=0.25in]{geometry}
\graphicspath{{./images/}}

\begin{document}

\begin{titlepage}
\title{Reinforcement learning gebaseerde agent voor presidenten}
\author{Freya Van Speybroek \& Thor Dossche}
\date{Academiejaar 2020 \- 2021}
\maketitle
\thispagestyle{empty}
\end{titlepage}


\section{Inleiding}
In dit project zullen we bestuderen hoe reinforcement learning kan toegepast worden op een kaartspel. Het kaartspel dat we gaan bekijken is presidenten. Dit is een perfect voorbeeld van een spel waarbij we incomplete informatie hebben: er kan namelijk niet in de kaarten van de tegenstanders gekeken worden. Daarbij is elke toestand discreet, waardoor het probleem kan gemodelleerd worden als een Partially Observable Markov Decision Process (POMDP).\\\\
Voor dit soort problemen zijn er al heel wat oplossingsmethodes, waarvan we er een paar zullen uitproberen en vergelijken. Daarbij kunnen we ons ook de vraag stellen of het wel degelijk rendeert om Reinforcement Learning te gebruiken, misschien zijn we beter af met een heristiek? 

\subsection{Het probleem}
Presidenten kent veel varianten, de regels liggen namelijk niet hard vast en kunnen daarom vaak besproken worden onder de spelers. Afhankelijk van welke regels je gebruikt, kan het spel moeilijker worden. \\\\
Bij het spelen van presidenten speel je meerdere spelletjes. Na ieder spel zullen de `rangen' voor het volgend spel gekend zijn. Ieder spel bestaat uit meerdere rondes. In deze rondes proberen de spelers zoveel mogelijk kaarten te leggen om zo al hun kaarten uit te spelen. Tijdens een ronde moeten spelers elk om beurt een kaart leggen hoger of gelijk aan de vorige kaart. Het aantal van die kaarten moet hoger of gelijk zijn aan de kaart(en) van de vorige beurt.\\
Als een speler geen kaart(en) meer kan of wil liggen past hij, eenmaal gepast zal hij niet meer aan de beurt komen in de ronde. Als alle spelers op een na gepast hebben is de winnaar van de ronde gekend. Deze speler mag een nieuwe ronde starten met een kaart of kaarten naar keuze. Als een speler al zijn kaarten heeft uitgespeeld krijgt hij een rang toegewezen voor het volgende spel.\\
De persoon die het eerste uitspeelt is de president, de tweede de vice-president.
De persoon die als laatste overblijft is de scum, degene die als voorlaatste uitspeelt is de vice-scum.\\
Als alle rangen gekend zijn start het volgende spel. Na het delen van de kaarten krijgt de president de beste 2 kaarten van de scum, de scum krijgt de 2 slechtste van de president. Hetzelfde geldt voor vice-president en vice-scum, maar die wisselen slechts 1 kaart uit.\\\\
Nog enkele basisregels:\\\\
- De speler die de ‘klaver 3’ heeft, start de eerste ronde van een spel met deze kaart (of meerdere kaarten van waarde 3 waaronder de `klaver 3'). \\
- Als een speler uit is, dan wordt een nieuwe ronde gestart en mag de speler links van de net uitgespeelde speler beginnen.\\\\
Om het spel interessanter te maken hebben wij nog een paar extra regels toegevoegd:\\\\
- Als een 7 is gespeeld, dan moet de volgende kaart lager of gelijk zijn aan die 7 ook het aantal kaarten moet gelijk zijn aan het aantal gelegde kaarten.\\
- De kaart 2 is een joker die kan gebruikt worden als hoogste kaart in het spel, of als een extra kaart. Als de joker gelegd wordt samen met een kaart van een andere waarde zal de joker deze waarde aannemen.\\
- De ranking van de kaarten is dus als volgt: 3,4,5,6,7,8,9,10,V,Q,K,A,2
\\\\

\subsection{De data}
Bij Reinforcement Learning gaan we niet aan de slag met een dataset, maar met een “environment”, die het spel zelf voorstelt. We willen dan een agent bekomen die een zo hoog mogelijke score behaalt in de environment. \\\\ 
In dit geval, willen we een agent die zo vaak mogelijk of zo lang mogelijk president kan zijn. Maar, de situatie is niet zo zwart-wit. Het kan namelijk ook goed zijn om veel vice-president te zijn, of om met heel slechte kaarten \- en dus met wat ongeluk \- toch geen scum te worden. Zou zelfs nooit scum worden kunnen gezien worden als een goed resultaat?  Het is niet zo dat 1 uitkomst goed is, en alle andere slecht. \\\\ 
Bovendien kan de environment ook op veel verschillende manieren bekeken worden. Hoe wordt een state voorgesteld, welke rewards delen we uit, welke acties maken we mogelijk? Dit zijn allemaal deelproblemen op zich, waarvoor we verschillende oplossingen zullen proberen vinden.\\\\
\end{document}
